\documentclass[twoside,11pt]{article}

% Any additional packages needed should be included after jmlr2e.
% Note that jmlr2e.sty includes epsfig, amssymb, natbib and graphicx,
% and defines many common macros, such as 'proof' and 'example'.
%
% It also sets the bibliographystyle to plainnat; for more information on
% natbib citation styles, see the natbib documentation, a copy of which
% is archived at http://www.jmlr.org/format/natbib.pdf

\usepackage{jmlr2e}
%\usepackage{parskip}

% Definitions of handy macros can go here
\newcommand{\dataset}{{\cal D}}
\newcommand{\fracpartial}[2]{\frac{\partial #1}{\partial  #2}}
% Heading arguments are {volume}{year}{pages}{submitted}{published}{author-full-names}

% Short headings should be running head and authors last names
\ShortHeadings{95-845: MLHC Proposal}{Lastname and Lastname}
\firstpageno{1}

\begin{document}

\title{Heinz 95-845: Project Proposal}

\author{\name Firstname Lastname \email ANDREWID/email@address.edu \\
       \addr Heinz College\\
       Carnegie Mellon University\\
       Pittsburgh, PA, United States} 

\maketitle


\section{Project Details}
Your project will involve the use of machine learning to conduct an analysis for a health care problem. This is an opportunity for you to explore some interest you have at the intersection of health care and machine learning.

The purpose of the project is to conduct an analysis that is novel in some way. The novelty could be in terms of development of machine learning, the assessment of a wide variety of machine learning algorithms at a focused health task, or the application of a single machine learning algorithm that solves a real health problem.

A list of exemplary papers are available in the Possible Data Sets slides on BlackBoard. The examples may be helpful in identifying how you conduct your study and prepare your write-up.

The proposal for the project is due on March 22nd. Please use this TeX template in Section \ref{details} and submit it either as a git repository (preferred) or by email to \url{jeremyweiss@cmu.edu}.

\subsection{Objectives}
The objective of this project proposal is to generate a proposal for your course project. It should be concise and describe the following components:
\begin{itemize}
\item Construction and description of a research framework that motivates the use of machine learning for your health-related task
\item Presentation of machine learning techniques appropriate for the task
\item Description of the data
\item Description of possible limitations of the study
\item Description of the likely analysis outcomes and their impact.
\end{itemize}

\subsection{Parameters}
The project is intended to be conducted in pairs, with one person taking the lead, and the second person able to provide input and feedback on the development. The second person will need to be able to run the first person's code, verify the main result/results, and write a short response, \emph{i.e.} a review of the first person's work. The second person will be added \emph{after} the proposals are turned in. \textbf{If your project cannot be reviewed by a second person, because of data privacy, computational frameworks, or other reasons, please document this in your proposal write-up}. In this case, you will be the second person on your project, and due to the overlap in content, your individual project grade will be worth correspondingly more by 5 points.

The project you propose should be different from an existing analysis either in the literature or from other class projects of yours. It is permissible to perform an analysis in data that warrants a second analysis. My guideline here is that the analysis must be greater than 50\% new. To get approval for these study, please describe the existing project that highlights the difference and contribution of this class's project, and provide any relevant documents (proposals, manuscripts, and/or citations). If the project has overlap with work from another course, you must also provide documented approval from the other faculty member/research collaborator(s). 

You are free to use programming language(s) of your choosing.

\section{Proposal Details (10 points)} \label{details}
Please provide information for the following fields. Your proposal write-up should be less than 2 pages.

\subsection{What is your proposed analysis? What are the likely outcomes?}


\subsection{Why is your proposed analysis important?}


\subsection{How will your analysis contribute to existing work? Provide references.}


\subsection{Describe the data. Please also define Y outcome(s), U treatment, V covariates, W population as applicable.}


\subsection{What evaluation measures are appropriate for the analysis? Which measures will you use?}


\subsection{What study design, pre-processing, and machine learning methods do you intend to use? Justify that the analysis is of appropriate size for a course project.}


\subsection{What are possible limitations of the study?}


\bibliography{}
%\appendix
%\section*{Appendix A.}
%Some more details about those methods, so we can actually reproduce them.

\end{document}
